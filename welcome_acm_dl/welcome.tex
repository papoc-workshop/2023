\documentclass[sigplan,review,anonymous]{acmart}
\renewcommand\footnotetextcopyrightpermission[1]{}
% Optional: Remove the ACM reference between the abstract and the main text.
\settopmatter{printfolios=true,printacmref=false}
% Optional: Comment out the CCS concepts and keywords.

\setcopyright{none}
\pagenumbering{gobble}

\usepackage{libertine}

\begin{document}

\title{Welcome}
\author{Elisa Gonzalez Boix}
\author{Pierre Sutra}
\maketitle
\vspace{-0.5cm}

The 10th Workshop on Principles and Practice of Consistency for Distributed Data (PaPoC) will take place on May 8th, 2023, in conjunction with EuroSys 2023.

\section*{Call for Papers}

Consistency is one of the fundamental issues of distributed computing. 
Beyond the well-known tension between Consistency, Availability, and Partition-tolerance, as captured by the CAP theorem, many nuanced consistency models and algorithms have been developed for different purposes. 
These consistency models have subtly different behaviour in practice, which translates to difficult choices between fault tolerance, performance, and programmability.
The issues and trade-offs are particularly vexing at scale, with a large number of processes or large shared databases, and in the presence of high latency and failure-prone networks, such as edge computing and peer-to-peer networks.

Since its inception in 2014, the PaPoC workshop series has brought together researchers and practitioners who seek to develop better techniques and a better understanding of consistency in distributed systems. 
We welcome contributions from a wide range of backgrounds: system development, distributed algorithms, concurrency, fault tolerance, databases, programming languages, blockchain, and verification. 
While there is no one universally best solution, we believe that by bringing together these perspectives, we can develop techniques that provide useful guarantees to applications, that are usable by application developers, and that satisfy real-world scalability, performance, and reliability requirements.

The workshop is looking for contributions on the following, and associated, topics:

\begin{itemize}
\item Techniques for scaling and improving the performance of strongly consistent systems (e.g., Paxos-like algorithms, state-machine replication protocols and distributed transactional systems).
\item Techniques for weak and hybrid consistency (such as session guarantees, causal consistency, operational transformation, conflict-free replicated data types (CRDTs), invariant-preserving replicated data types, monotonic programming, state merging, operation commutativity, etc).
\item Data consistency in geo-replicated, peer-to-peer, and edge computing systems.
\item How to expose consistency vs. performance and scalability trade-offs in the programming model, and how to help developers choose.
\item How to support composed operations spanning multiple objects (transactions, sagas, workflows).
\item Techniques or tools to aid the development of replicated data (e.g., reasoning, analysis and verification of application programs using storage systems with various consistency models, visualization techniques for distributed dependencies or state merges, etc.).
\item Formal methods for distributed systems dealing with strong/weak consistent data (such as techniques for verifying safety, liveness or consistency properties, convergence verification, etc.) 
\item Implementation techniques and optimisations for replicated data types to improve fault tolerance, security, application-level invariants, metadata usage, and controlling divergence.
\item Studies of performance, scalability, and programmability for the aforementioned systems.
\end{itemize}
  
\section*{Program committee chairs}

\begin{itemize}
\item Elisa Gonzalez Boix, Vrije Universiteit Brussel, Belgium
\item Pierre Sutra, Telecom SudParis, France
\end{itemize}

\section*{Program committee}

\begin{itemize}
\item Valter Balegas, Electric-SQL
\item Annette Bieniusa, University of Kaiserslautern-Landau, Germany
\item Vitor Enes, Teleport
\item Carla Ferreira, NOVA University of Lisbon, Portugal
\item Gowtham Kaki, University of Colorado, Boulder, USA
\item Lindsey Kuper, University of California, USA
\item Xiaojian Liao, Tsinghua University, China
\item Ragnar Mogk, TU Darmstadt, Germany
\item Shuai Mu, Stony Brook University, USA
\item Kartik Nagar, Indian Institute of Technology, Madras, India
\item Sreeja Nair, ZettaScale
\item Fernando Pedone, University of Lugano, Switzerland
\item Mathieu Perrin, University of Nantes, France
\item Jack Waudby, Neo4j
\end{itemize}

%% \section*{Steering committee}

%% \begin{itemize}
%% \item Peter Bailis, Stanford University, USA
%% \item Carlos Baquero, HASLab, INESC TEC and University of Minho, Portugal
%% \item Annette Bieniusa, University of Kaiserslautern, Germany
%% \item Carla Ferreira, Universidade NOVA de Lisboa, Portugal
%% \item Alexey Gotsman, IMDEA Software Institute, Spain
%% \item Martin Kleppmann, Cambridge University, UK
%% \item Heather Miller, Carnegie Mellon University, USA
%% \item Nuno Preguiça, NOVA-LINCS and NOVA University of Lisbon, Portugal
%% \item Marco Serafini, University of Massachusetts Amherst, USA
%% \item Marc Shapiro, Sorbonne-Universités—LIP6 and Inria, France
%% \item Justin Sheehy, Akamai Technologies, USA
%% \end{itemize}


\end{document}
