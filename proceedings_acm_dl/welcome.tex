\documentclass[acmlarge,nonacm]{acmart}
\usepackage{quoting}
\usepackage{ragged2e}

\pdfpagewidth=8.5in
\pdfpageheight=11in

\settopmatter{printfolios=false}

\newenvironment{RCText}[1][2em]
               {\begin{quoting}[leftmargin=#1,rightmargin=.5cm]\RaggedRight}
               {\end{quoting}}

\begin{document}

\title{Welcome}
\author{Elisa Gonzalez Boix}
\author{Pierre Sutra}
\authorsaddresses{}
\renewcommand{\shortauthors}{}
\maketitle

The 10th Workshop on Principles and Practice of Consistency for Distributed Data (PaPoC '23) took place on May 8th, 2023.
Like the previous years, the PaPoC workshop was held in conjunction with the EuroSys conference.
A total of 14 papers where presented, 12 of them are in the proceedings.
These papers were selected among 20 papers which were initially submitted to the workshop.
In addition to the accepted papers, the workshop had also an invited talk by Vijay Chidambaram, University of Texas, Austin, USA.

This year edition was the tenth occurrence of the PaPoC workshop series.
To celebrate this event, the organizers, in agreement with the Steering Committee, decided to give a Test-Of-Time award to an influential paper that was published over the past 10 years at the workshop.
A Test-Of-Time Award Committee was formed for this occasion.
The \href{http://www.bailis.org/blog/bridging-the-gap-opportunities-in-coordination-avoiding-databases/}{paper} they selected is:

\medskip

\begin{RCText}[2cm]
  \textit{Bridging the Gap: Opportunities in Coordination-Avoiding Database Systems} \\
  by Peter Bailis, Alan Fekete, Ali Ghodsi, Mike Franklin, Joe Hellerstein, and Ion Stoica.
\end{RCText}

\medskip

This paper was presented at PaPoC 2014, held in conjunction with the Eurosys '14 conference, in Amsterdam, Netherlands.
This work was influential in challenging the common idea of using strong consistency levels as a one-size-fits-all solution.
In this paper, the authors look at popular benchmarks and applications and investigate where coordination is actually needed, and where it can be avoided.
An extended version of this paper was later presented at SIGMOD in 2014.

The composition of the Program Committee and Test-of-Time Award Committee for PaPoc '23 are below.
The Steering Committee for the PaPoC workshop series is also listed.

\section*{Program Committee}

\begin{itemize}
\item Valter Balegas, Electric-SQL
\item Annette Bieniusa, University of Kaiserslautern-Landau, Germany
\item Vitor Enes, Teleport
\item Carla Ferreira, NOVA University of Lisbon, Portugal
\item Gowtham Kaki, University of Colorado, Boulder, USA
\item Lindsey Kuper, University of California, Santa Cruz, USA
\item Xiaojian Liao, Tsinghua University, China
\item Ragnar Mogk, TU Darmstadt, Germany
\item Shuai Mu, Stony Brook University, USA
\item Kartik Nagar, Indian Institute of Technology, Madras, India
\item Sreeja Nair, ZettaScale
\item Fernando Pedone, University of Lugano, Switzerland
\item Mathieu Perrin, University of Nantes, France
\item Jack Waudby, Neo4j
\end{itemize}

\section*{Test-of-Time Award Committee}

\begin{itemize}
\item Alysson Bessani, University of Lisbon, Portugal
\item Roberto Palmieri, Lehigh University, USA
\item Marco Serafini, University of Massachusetts, Amherst, USA
\item Marko Vukolic, Protocol Labs
\end{itemize}

\section*{Steering Committee}

\begin{itemize}
\item Peter Bailis, Sisu Data
\item Carlos Baquero, HASLab, INESC TEC and University of Minho, Portugal
\item Annette Bieniusa, University of Kaiserslautern-Landau, Germany
\item Carla Ferreira, NOVA University of Lisbon, Portugal
\item Alexey Gotsman, IMDEA Software Institute, Spain
\item Martin Kleppmann, Cambridge University, UK
\item Heather Miller, Carnegie Mellon University, USA
\item Nuno Preguiça, NOVA-LINCS and NOVA University of Lisbon, Portugal
\item Marco Serafini, University of Massachusetts, Amherst, USA
\item Marc Shapiro, Sorbonne-Universités—LIP6 and Inria, France
\item Justin Sheehy, Akamai Technologies
\end{itemize}

\end{document}
