\documentclass[screen,11pt]{acmart}

\setcopyright{none}
\pagenumbering{gobble}

\usepackage{libertine}

\begin{document}
\title{Welcome}
\author{Adriana Szekeres}
\author{KC Sivaramakrishnan}
\maketitle
\vspace{-0.5cm}

The 9th workshop on the Principles and Practice of Consistency (PaPoC) for
Distributed Data will be held in Rennes, France on April 5th, 2022. The PaPoC
workshop investigates the trade-offs among different consistency models for
distributed systems, and their operational characteristics. PaPoC 2022 follows
the successful workshops in this series and brings together researchers and
practitioners in the areas of distributed systems, programming languages,
databases, and concurrent programming.

Just like the previous years, PaPoC 2022 will be co-located with EuroSys 2022.
As the world is slowly opening up after the COVID-19 pandemic, the workshop
will be held in hybrid mode with remote and in-person speakers and
participants.

The workshop program was selected by an international committee of experts from
the industry and academia. Each paper received three reviews from the members
of the program committee. The quality of the submissions was quite high. The
workshop received 12 submission and out of these 11 papers were selected for
the workshop. In addition to the accepted papers, workshop will also have two
invited talks one from the industry (Karthik Ranganathan, CTO, YugaByte) and
another from academia (Joe Hellerstein, Professor, UC Berkeley).

The details of the organizing committee follows:

\subsection*{Program committee chairs}

\begin{itemize}
\item Adriana Szekeres, VMware Research, USA
\item KC Sivaramakrishnan, Indian Institute of Technology, Madras, India
\end{itemize}

\subsection*{Program committee}

\begin{itemize}

\item Hagit Attiya, Technion, Israel
\item Sebastian Burckhardt, Microsoft Research, USA
\item Vasilis Gavrielatos, Huawei Research UK, UK
\item Alexey Gotsman, IMDEA Software Institute, Spain
\item Gowtham Kaki, University of Colorado, Boulder, USA
\item Akash Lal, Microsoft Research, India
\item João Leitão, NOVA-LINCS and NOVA University of Lisbon, Portugal
\item Jialin Li, NUS, Singapore
\item Hongjin Liang, Nanjing University, China
\item Mae Milano, University of California, Berkeley, USA
\item Madhavan Mukund, Chennai Mathematical Institute, India
\item Kartik Nagar, Indian Institute of Technology, Madras, India
\item Sreeja Nair, ADLINK Technology, France
\item Naveen Sharma, Google, USA

\end{itemize}

\subsection*{Steering committee}

\begin{itemize}
\item Peter Bailis, Stanford University, USA
\item Carlos Baquero, HASLab, INESC TEC and University of Minho, Portugal
\item Annette Bieniusa, University of Kaiserslautern, Germany
\item Carla Ferreira, Universidade NOVA de Lisboa, Portugal
\item Alexey Gotsman, IMDEA Software Institute, Spain
\item Martin Kleppmann, Cambridge University, UK
\item Heather Miller, Carnegie Mellon University, USA
\item Nuno Preguiça, NOVA-LINCS and NOVA University of Lisbon, Portugal
\item Marco Serafini, University of Massachusetts Amherst, USA
\item Marc Shapiro, Sorbonne-Universités—LIP6 and Inria, France
\item Justin Sheehy, Akamai Technologies, USA
\end{itemize}


\end{document}
